\section{Abstract homological algebra}
\epigraph{\textit{Tea is nought but this: first you heat the water, then you make tea. Then you drink it properly. That is all you need to know.}}{Sen No Rikyu}
\vspace{10 mm}
Before starting, let us fix some anti-Bourbaki categorical conventions. We'll often consider objects of a category up to isomorphism. This means, for example, that when an isomorphism is obvious or canonical we'll just write equality and that all subcategories are assumed to be saturated (i.e. closed under isomorphisms). To keep things clean, we will avoid writing names for morphisms as much as possible and will denote $1$ the identity of any object in any category. We do not really care about foundational issues and will almost always assume our categories to be (essentially) small. There will occasionaly be lines preceeded by the \twonotes \ symbol. Those will be just vague streams of consciousness and are mainly meant as decorations. \\

\subsection{Triangulated categories}
We begin by introducing the abstract framework in which homological algebra usually takes place: the so called triangulated categories. These are immediate generalizations of derived categories which do not depend on an underlying abelian category. Indeed, one just has a notion of 'distinguished triangle', which we can think of as some kind of short exact sequence, and an autoequivalence which corresponds to shifting the indices of a complex in the derived category. By attaching a distinguished triangle to its shifts one builds up the idea of induced long exact sequence and this is the essential tool of homological algebra. \\

Triangulated categories first appeared in Jean-Louis Verdier's PhD thesis supervised by Alexander Grothendieck (\cite{ver}), still a solid reference today. A more modern and ramified treatment can be found in the classical textbook \cite{gel}. However, we give a slightly different presentation: following \cite{may}, one of the usual axioms for triangulated categories results redundant and we prove it as a proposition accordingly. On the other hand, as it had been for many years with the fifth postulate in euclidean geometry and with the axiom of choice in set theory, it is not known whether \textit{(Tr4)} below follows from the other axioms. 

\begin{defn}
Let $\mathscr{D}$ be an additive category, $\Sigma$ an additive autoequivalence of $\mathscr{D}$. We denote for all $n \in \mathbb{Z}$, $*[n]=\Sigma^n(*)$. A \textbf{triangle} in $\mathscr{D}$ is a diagram of the form $X \longrightarrow Y \longrightarrow Z \longrightarrow X[1]=\Sigma(X)$, which we denote for simplicity: $$X \longrightarrow Y \longrightarrow Z \longrightarrow$$
The triangles of $\mathscr{D}$ form then an additive category, where the morphisms are triples $a,b,c$ fitting in a commutative diagram: 
\begin{center}
\begin{tikzcd}[ampersand replacement=\&]
X \arrow{r} \arrow{d}{a} \& Y \arrow{r} \arrow{d}{b} \& Z \arrow{d}{c} \arrow{r} \& X[1] \arrow{d}{a[1]} \\
X' \arrow{r} \& Y' \arrow{r} \& Z' \arrow{r} \& X'[1]
\end{tikzcd}
\end{center}
A \textbf{triangulated category} is an additive category $\mathscr{D}$ equipped with an additive autoequivalence $\Sigma$, called \textbf{translation}, and a full (saturated) subcategory of its triangles $\bigtriangleup$, whose objects we call \textbf{distinguished triangles} of $\mathscr{D}$, so that: 
\begin{enumerate}
\item [(Tr1)] The triangle $$X \overset{1}{\longrightarrow} X \longrightarrow 0 \longrightarrow$$ is distinguished. We call it the \textbf{trivial} triangle.
\item [(Tr2)] Any morphism $X \longrightarrow Y$ in $\mathscr{D}$ extends to a distinguished triangle $X \longrightarrow Y \longrightarrow Z \longrightarrow $. We call $Z$ a \textbf{cone} of that morphism. 
\item [(Tr3)] The triangle $X \overset{f}{\longrightarrow} Y \longrightarrow Z \longrightarrow $ is distinguished if and only if the triangle $$Y \longrightarrow Z \longrightarrow  X[1]  \overset{-f[1]}{\longrightarrow} $$ is distinguished. We call the second triangle the \textbf{rotation} of the first one.
\item [(Tr4)] If  $X \overset{f}{\longrightarrow} Y \overset{h}{\longrightarrow} Z' \longrightarrow $, $Y \overset{g}{\longrightarrow} Z \longrightarrow X' \longrightarrow$ and $X \overset{gf}{\longrightarrow} Z \longrightarrow Y'$ are distinguished triangles, then there is a distinguished triangle $Z' \longrightarrow Y' \longrightarrow X' \longrightarrow $, called a \textbf{braid} generated by $gf$, so that the following commutes: \\
\begin{center}
\scalebox{0.8}{
\begin{tikzcd}[ampersand replacement=\&]
  X \arrow[bend left=35]{rr}{gf} \arrow{rd}{f} \& \&  Z \arrow[bend left=35]{rr} \arrow{rd} \& \& X' \arrow[bend left=35]{rr} \arrow{rd} \& \& Z'[1] \\
 \& Y \arrow{ru}{g} \arrow{rd}{h} \& \& Y' \arrow{ru} \arrow{rd} \& \& Y[1] \arrow{ru}{h[1]} \& \\
 \& \& Z' \arrow{ru} \arrow[bend right=35]{rr} \& \& X[1] \arrow{ru}{f[1]} \& \&
\end{tikzcd}
}
\end{center}
\end{enumerate}
\end{defn}

Let $\mathscr{D}$ be a triangulated category. Observe that the opposite category $\mathscr{D}^{\textnormal{op}}$ equipped with translation functor $\Sigma^{-1}$ is clearly triangulated.\\
We put the sign in \textit{(Tr3)} because, if $X \overset{f}{\longrightarrow} Y \overset{g}{\longrightarrow} Z \overset{h}{\longrightarrow} $ is a distinguished triangle in $\mathscr{D}$, each three consecutive morphisms in the following long sequence form a distinguished triangle: $$\cdots \overset{-g[-1]}{\longrightarrow} Z[-1] \overset{-h[-1]}{\longrightarrow} X \overset{f}{\longrightarrow} Y \overset{g}{\longrightarrow} Z \overset{h}{\longrightarrow} X[1] \overset{-f[1]}{\longrightarrow} Y[1] \overset{-g[1]}{\longrightarrow} \cdots$$ 

\begin{exmp}\label{bbe}
Let $\mathscr{A}$ be an abelian category. Recall that its derived category $\mathscr{D}(\mathscr{A})$ is the homotopy category of chain complexes in $\mathscr{A}$ localized to quasi-isomorphisms (i.e. morphisms which induce isomorphisms in homology). The translation functor of $\mathscr{D}(\mathscr{A})$ shifts the indices of a complex and changes the sign of the differential. Moreover, the cone $\textnormal{cone}(f)$ of a morphism $f$ of complexes is the total complex of $f$ (seen as a double complex). If we take as distinguished triangles in $\mathscr{D}(\mathscr{A})$ those isomorphic to triangles of the form $$X \overset{f}{\longrightarrow} Y \longrightarrow \textnormal{cone}(f) \longrightarrow$$
the derived category becomes tiangulated. Moreover, the bounded derived category $\mathscr{D}^b(\mathscr{A})$ (consisting of complexes with homology vanishing for a cofinite set of indices) is a triangulated subcategory of $\mathscr{D}(\mathscr{A})$. 
\end{exmp}

\begin{prop}\label{s}
\textnormal{($3 \times 3$ Lemma)} Suppose that $X \overset{f}{\longrightarrow} Y \overset{g}{\longrightarrow} Z \overset{h}{\longrightarrow}$, $X' \overset{f'}{\longrightarrow} Y' \longrightarrow Z' \longrightarrow $, $X \overset{a}{\longrightarrow} X' \overset{a'}{\longrightarrow} X'' \overset{a''}{\longrightarrow} $, $Y \overset{b}{\longrightarrow} Y' \longrightarrow Y'' \longrightarrow $ are distinguished triangles in $\mathscr{D}$. If the upper left square of the below diagram commutes, then there are distinguished triangles $Z \longrightarrow Z' \longrightarrow Z'' \longrightarrow $, $X'' \longrightarrow Y'' \longrightarrow Z'' \longrightarrow $ so that the following commutes:  
\begin{center}
\begin{tikzcd}[ampersand replacement=\&]
X \arrow{r}{f} \arrow{d}{a} \& Y \arrow{r}{g} \arrow{d}{b} \& Z \arrow{r}{h} \arrow{d} \& X[1] \arrow{d}{a[1]} \\
X' \arrow{r}{f'} \arrow{d}{a'} \& Y' \arrow{r} \arrow{d} \& Z' \arrow{r} \arrow{d} \& X'[1] \arrow{d}{a'[1]} \\
X'' \arrow{r} \arrow{d}{a''} \& Y'' \arrow{r} \arrow{d} \& Z'' \arrow{r} \arrow{d} \& X''[1] \arrow{d}{-a''[1]} \\
X[1] \arrow{r}{f[1]} \& Y[1] \arrow{r}{g[1]} \& Z[1] \arrow{r}{h[1]} \& X[2] 
\end{tikzcd}
\end{center}
(caution: the bottom row may not be distinguished).
\end{prop}

\begin{proof}
By taking a cone of $bf=f'a$, we have a distinguished triangle $X \overset{bf}{\longrightarrow} Y' \longrightarrow V \longrightarrow $. By taking braids generated by $bf$ and $f'a$ respectively, we have then two distinguished triangles: $$Z \overset{s}{\longrightarrow} V \overset{t}{\longrightarrow} Y'' \longrightarrow$$   $$X'' \overset{s'}{\longrightarrow} V \overset{t'}{\longrightarrow} Z' \longrightarrow$$ 
By taking a cone of $ts'$, we get a distinguished triangle  $$X'' \longrightarrow Y'' \longrightarrow Z'' \longrightarrow$$ 
which is one of the triangles of the thesis. \\
Now, by rotation we get a distinshed triangle $V \overset{t}{\longrightarrow} Y'' \longrightarrow Z[1] \overset{-s[1]}{\longrightarrow}$ and by taking a braid generated by $ts'$ we have another distinguished triangle: 
$$Z' \longrightarrow Z'' \longrightarrow Z[1] \longrightarrow $$
Rotating this last triangle, we have the other triangle of the thesis. \\
The commutativity follows by the way we applied \textit{(Tr4)}. 
\end{proof}

\begin{prop}\label{a}
Let $X \longrightarrow Y \longrightarrow Z \longrightarrow$ and $X' \longrightarrow Y' \longrightarrow Z' \longrightarrow$ distinguished triangles of $\mathscr{D}$, $Y \overset{b}{\longrightarrow} Y'$ a morphism. Then the following are equivalent:  \\
\begin{enumerate}
\item there is a morphism $X \overset{a}{\longrightarrow} X'$ so that the first square of the below diagram commutes  
\item there is a morphism $Z \overset{c}{\longrightarrow} Z'$ so that the second square of the below diagram commutes  
\item $b$ extends to a morphism between the two triangles:
\end{enumerate} 
\begin{center}
\begin{tikzcd}[ampersand replacement=\&]
X \arrow{r} \arrow{d}{a} \& Y \arrow{r} \arrow{d}{b} \& Z \arrow{d}{c} \arrow{r} \& X[1] \arrow{d}{a[1]} \\
X' \arrow{r} \& Y' \arrow{r} \& Z' \arrow{r} \& X'[1]
\end{tikzcd}
\end{center}
\end{prop}

\begin{proof}
Clearly, \textit{(3)} implies the other ones. To show that \textit{(1)} implies \textit{(3)}, take cones of $a$ and $b$ and apply the $3 \times 3$ Lemma. But rotating the triangles, we see by the same argument that \textit{(2)} implies \textit{(3)}. 
\end{proof}

The above Proposition means that we can start with two among three of the morphisms involved in a morphisms of triangles and obtain the third. We will refer to this opertation as \textbf{completing} the two morphisms to a morphism of triangles. \\

\begin{prop}\label{b}
The composition of any two consecutive morphisms of a distinguished triangle is $0$.
\end{prop}

\begin{proof}
Let $X \overset{f}{\longrightarrow} Y \overset{g}{\longrightarrow} Z \longrightarrow$ be a distinguished triangle. Then complete $f$ and the identity of $X$ to a morphism of of triangles between that triangle and the trivial one: 
\begin{center}
\begin{tikzcd}[ampersand replacement=\&]
X \arrow{r}{1} \arrow{d}{1} \& X \arrow{r} \arrow{d}{f} \& 0 \arrow{d} \arrow{r} \& X[1] \arrow{d}{1} \\
X \arrow{r}{f} \& Y \arrow{r}{g} \& Z \arrow{r} \& X[1]
\end{tikzcd}
\end{center}
That shows $gf=0$. For the other cases, just rotate the triangle.  
\end{proof}

\begin{defn}
Let $\mathscr{A}$ be an abelian category. A functor $\mathscr{D} \overset{H}{\longrightarrow} \mathscr{A}$ is \textbf{cohomological} if for each distinguished triangle $X \longrightarrow Y \longrightarrow Z \longrightarrow$ of $\mathscr{D}$ its image $$H(X) \longrightarrow H(Y) \longrightarrow H(Z)$$ 
is exact in $\mathscr{A}$. 
\end{defn}

\begin{exmp}
Using the spectral sequence of a double complex one sees that the $i$-th homology functor $\mathscr{D}(\mathscr{A}) \overset{H^i}{\longrightarrow} \mathscr{A}$ is cohomological for each $i \in \mathbb{Z}$. This will be anyway proved in a general context in the next section. 
\end{exmp}

\begin{prop}
For each object $E \in \mathscr{D}$, the functors $\textnormal{Hom}_{\mathscr{D}}(E, *)$ and $\textnormal{Hom}_{\mathscr{D}}(*,E)$ are cohomological (with values in $\textnormal{Mod}_{\mathbb{Z}}$ and $\textnormal{Mod}^{\textnormal{op}}_{\mathbb{Z}}$ respectively). 
\end{prop}

\begin{proof}
We'll just show that the first functor is cohomological, for the other one the argument is dual. Let $X \overset{f}{\longrightarrow} Y \overset{g}{\longrightarrow} Z \longrightarrow$ be a distinguished triangle, $E \overset{a}{\longrightarrow} Y$ a morphism so that $ga=0$. Complete $a$ and $0 \longrightarrow Z$: 
\begin{center}
\begin{tikzcd}[ampersand replacement=\&]
E \arrow{r}{1} \arrow{d}{b} \& E \arrow{r} \arrow{d}{a} \& 0 \arrow{d} \arrow{r} \& E[1] \arrow{d}{b[1]} \\
X \arrow{r}{f} \& Y \arrow{r}{g} \& Z \arrow{r} \& X[1]
\end{tikzcd}
\end{center}
This shows that, for some $b$, $fb=a$, as desired. 
\end{proof}

\begin{defn}
A triangle $X \longrightarrow Y \longrightarrow Z  \longrightarrow$ in $\mathscr{D}$ (not necessarily distinguished) is \textbf{special} if for each $E \in \mathscr{D}$ the induced long sequence of abelian groups:
\begin{center}
\scalebox{0.9}{
\begin{tikzcd}[ampersand replacement=\&]
 \cdots \longrightarrow \textnormal{Hom}_{\mathscr{D}}(E,X) \arrow{r} \& \textnormal{Hom}_{\mathscr{D}}(E,Y) \arrow{r} \ar[draw=none]{d}[name=X, anchor=center]{} \&  \textnormal{Hom}_{\mathscr{D}}(E,Z) \ar[rounded corners, to path={ -- ([xshift=2ex]\tikztostart.east) |- (X.center) \tikztonodes -| ([xshift=-2ex]\tikztotarget.west) -- (\tikztotarget)}]{dll}[at end]{} \\
 \textnormal{Hom}_{\mathscr{D}}(E,X[1]) \arrow{r} \& \textnormal{Hom}_{\mathscr{D}}(E,Y[1]) \arrow{r} \& \textnormal{Hom}_{\mathscr{D}}(E,Z[1]) \longrightarrow \cdots 
\end{tikzcd}
}
\end{center}
is exact.
\end{defn}

To be clear, saying that $\textnormal{Hom}_{\mathscr{D}}(E,*)$ is cohomological means that distingusihed triangles are special. The converse is not true in general: if we change the sign of one of the morphisms in a distinguished triangle we obtain a special triangle which doesn't have to be distinguished.\\

\begin{prop}\label{c}
Let 
\begin{center}
\begin{tikzcd}[ampersand replacement=\&]
X \arrow{r} \arrow{d}{a} \& Y \arrow{r} \arrow{d}{b} \& Z \arrow{d}{c} \arrow{r} \& X[1] \arrow{d}{a[1]} \\
X' \arrow{r} \& Y' \arrow{r} \& Z' \arrow{r} \& X'[1]
\end{tikzcd}
\end{center}
be a morphism of special triangles. If two among $a,b,c$ are isomorphisms, then so is the third. 
\end{prop}

\begin{proof}
As usual, up to rotation, we'll show the statement in the case $a$ and $c$ are isomorphisms. For each $E \in \mathscr{D}$, we have the diagram: 
\begin{center}
\scalebox{0.75}{
\begin{tikzcd}[ampersand replacement=\&, column sep=small]
\textnormal{Hom}_{\mathscr{D}}(E,Z[-1]) \arrow{r} \arrow{d} \& \textnormal{Hom}_{\mathscr{D}}(E,X)  \arrow{r} \arrow{d} \& \textnormal{Hom}_{\mathscr{D}}(E,Y)  \arrow{d} \arrow{r} \& \textnormal{Hom}_{\mathscr{D}}(E,Z) \arrow{r} \arrow{d} \& \textnormal{Hom}_{\mathscr{D}}(E,X[1]) \arrow{d} \\
\textnormal{Hom}_{\mathscr{D}}(E,Z'[-1]) \arrow{r}  \& \textnormal{Hom}_{\mathscr{D}}(E,X')  \arrow{r}  \& \textnormal{Hom}_{\mathscr{D}}(E,Y')  \arrow{r} \& \textnormal{Hom}_{\mathscr{D}}(E,Z') \arrow{r}  \& \textnormal{Hom}_{\mathscr{D}}(E,X'[1]) \\
\end{tikzcd}
}
\end{center}
The rows are exact since the triangles are special, and the two left and two right vertical morphisms are isomorphisms of abelian groups by hypothesis. By the Five Lemma for abelian categories we conclude that the middle vertical morphism is an isomorphism, and since this is true for each $E \in \mathscr{D}$, by the Yoneda Lemma we conclude that $b$ is an isomorphism. 
\end{proof}
 
It follows immediately that the cone of a morphism $f$ in $\mathscr{D}$ is unique up to (in general not unique) isomorphism, so we denote it $\textnormal{cone}(f)$. Unfortunately, it is not even possible to make a functorial choice for cones (indeed, when it's possible, then $\mathscr{D}$ is semisimple abelian, as shown in \cite{ver}) and thus we will always assume that an arbitrary choice of cones has been made in $\mathscr{D}$. We will establich a uniqueness property in the following theorem, which is a stronger version of \hyperref[a]{\textbf{Proposition \ref*{a}}}.\\

\begin{prop}\label{g}
Let $X \overset{f}{\longrightarrow} Y \longrightarrow Z \longrightarrow$ and $X' \longrightarrow Y' \overset{g}{\longrightarrow} Z' \longrightarrow$ distinguished triangles of $\mathscr{D}$, $Y \overset{b}{\longrightarrow} Y'$ a morphism. Then $b$ extends to a morphism of triangles: 
\begin{center}
\begin{tikzcd}[ampersand replacement=\&]
X \arrow{r}{f} \arrow{d}{a} \& Y \arrow{r} \arrow{d}{b} \& Z \arrow{d}{c} \arrow{r} \& X[1] \arrow{d}{a[1]} \\
X' \arrow{r} \& Y' \arrow{r}{g} \& Z' \arrow{r} \& X'[1]
\end{tikzcd}
\end{center}
if and only if $gbf=0$. Moreover, if that's the case and $\textnormal{Hom}_{\mathscr{D}}(X,Z'[-1])=0$, then $a$ and $c$ are unique. 
\end{prop}

\begin{proof}
The 'only if' part follows from \hyperref[b]{\textbf{Proposition \ref*{b}}}. For the 'if' part, applying the cohomological functor $\textnormal{Hom}_{\mathscr{D}}(X,*)$ to the second triangle, we get an exact sequence of abelian groups: $$\textnormal{Hom}_{\mathscr{D}}(X,Z[-1]) \longrightarrow \textnormal{Hom}_{\mathscr{D}}(X,X') \longrightarrow \textnormal{Hom}_{\mathscr{D}}(X,Y') \longrightarrow \textnormal{Hom}_{\mathscr{D}}(X,Z')$$
Since $bf$ is in the kernel of the last map of the above sequence by hypothesis, it is in the image of the second map and we can then take $a$ in its preimage. If $\textnormal{Hom}_{\mathscr{D}}(X,Z'[-1])=0$, then $a$ is unique since the second map is injective. To get $c$, just apply \hyperref[a]{\textbf{Proposition \ref*{a}}}.   
\end{proof}

\begin{prop}\label{i}
Let $X \overset{f}{\longrightarrow} Y$ be a morphism in $\mathscr{D}$. Then $f$ is an isomorphism if and only if $\textnormal{cone}(f)=0$. 
\end{prop}

\begin{proof}
Consider the diagram and the notation of \hyperref[b]{\textbf{Proposition \ref*{b}}}. Since $1$ is an automorphism of $X$, by \hyperref[c]{\textbf{Proposition \ref*{c}}} $0 \longrightarrow Z=\textnormal{cone}(f)$ is an isomorphism if and only if $f$ is an isomorphism. 
\end{proof}

Now we deal with direct sums. First of all, the following proposition shows that the cone commutes with direct sums. 

\begin{prop}\label{e}
Two triangles $X \longrightarrow Y \longrightarrow Z \longrightarrow$ and $X' \longrightarrow Y' \longrightarrow Z' \longrightarrow$ are both distinguished if and only if the triangle $$X \oplus X' \longrightarrow Y \oplus Y' \longrightarrow Z \oplus Z' \longrightarrow$$ 
is distinguished (the maps of the third triangle are the direct sums of the maps of the first two triangles).
\end{prop}

\begin{proof}
Let's show the 'only if' part. Denote $Q=\textnormal{cone}(X \oplus X' \longrightarrow Y \oplus Y')$. By completing the inclusions of the summands in the direct sums, we get a morphisms $Z \longrightarrow Q$ and $Z' \longrightarrow Q$. Denote $c$ the direct sum of these two morphisms. We have a morphism of triangles: 
\begin{center}
\begin{tikzcd}[ampersand replacement=\&]
X \oplus X' \arrow{r} \arrow{d}{1} \& Y \oplus Y' \arrow{r} \arrow{d}{1} \& Z \oplus Z' \arrow{d}{c} \arrow{r} \& X[1] \oplus X'[1] \arrow{d}{1} \\
X \oplus X' \arrow{r} \&  Y \oplus Y' \arrow{r} \& Q \arrow{r} \&  X[1] \oplus X'[1]
\end{tikzcd}
\end{center}
The upper triangle is special because it is direct sum of special triangles, and thus $c$ is an isomorphism. \\
For the 'if' part, the argument is similar. Denote $Q=\textnormal{cone}(X \longrightarrow Y)$. Completing the projections to the summands of the direct sums, we have a morphism $Z \oplus Z' \longrightarrow Q$, and composing with the incluzion of $Z$, we get $Z \longrightarrow Q$. This is an isomorphism by the same argument as above, since direct summands of special triangles are special by an easy check. 
\end{proof}

In other words, the cone commutes with direct sums or, more precisely, $\bigtriangleup$ is an additive subcategory of triangles closed under direct summands. As a consequence, we have that for each $X,Y \in \mathscr{D}$ the triangle $$X \longrightarrow X \oplus Y \longrightarrow Y \overset{0}{\longrightarrow} $$ is distinguished (the maps are inclusion and projection), because it is the sum of a trivial and a rotated trivial triangle. Conversely, we have the following result. 

\begin{prop}\label{d}
Let $X \overset{f}{\longrightarrow} Y \overset{g}{\longrightarrow} Z \overset{h}{\longrightarrow}$ be a distingusihde triangle of $\mathscr{D}$. If $h=0$, then $g$ has a right inverse. Moreover, for every right inverse $s$ of $g$, $$X \oplus Z \overset{(f \ s)}{\longrightarrow} Y$$ is an isomorphism. 
\end{prop}

\begin{proof}
Applying the cohomological functor $\textnormal{Hom}_{\mathscr{D}}(Z,*)$ to the triangle, we see get an exact sequence of abelian groups: $$\textnormal{Hom}_{\mathscr{D}}(Z,Y) \longrightarrow \textnormal{Hom}_{\mathscr{D}}(Z,Z) \longrightarrow \textnormal{Hom}_{\mathscr{D}}(Z,X[1])$$ The surjectivity of the first map is equivalent to $g$ having right inverse and to $h=0$. Thus the first statement of the theorem is proven. Now suppose $g$ has right inverse, and thus $h=0$. For each $W \in \mathscr{D}$, by applying the cohomological functor $\textnormal{Hom}_{\mathscr{D}}(W,*)$ to the triangle we get a short exact sequence of abelian groups: $$0 \longrightarrow \textnormal{Hom}_{\mathscr{D}}(W,X) \longrightarrow \textnormal{Hom}_{\mathscr{D}}(W,Y) \longrightarrow \textnormal{Hom}_{\mathscr{D}}(W,Z) \longrightarrow 0$$
By the splitting lemma $\textnormal{Hom}_{\mathscr{D}}(W,Y)=\textnormal{Hom}_{\mathscr{D}}(W,X \oplus Z)$ and by arbitrariness of $W$ we get, using the Yoneda lemma, the desired result. 
\end{proof}

\begin{defn}
A \textbf{triangle functor} between two triangulated categories $\mathscr{D}$ and $\mathscr{D}'$ with translation functors $\Sigma$ and $\Sigma'$ respectively is a functor $\mathscr{D} \overset{F}{\longrightarrow} \mathscr{D}'$ equipped with an isomorphism of functors $\{ \xi_X \}_{X \in \mathscr{D}}$ between $F \Sigma$ and $\Sigma'F$ so that for each distinguished triangle $X \longrightarrow Y \longrightarrow Z \overset{h}{\longrightarrow}$ in $\mathscr{D}$, the triangle $$F(X) \longrightarrow F(Y) \longrightarrow F(Z) \overset{\xi_X F(h)}{\longrightarrow}$$ is distinguished in $\mathscr{D}'$. \\
We denote $\textnormal{Aut}(\mathscr{D})$ the group of triangle autoequivalences of $\mathscr{D}$. 
\end{defn}

The class of triangulated categories has then a structure of a $2$-category, whose $1$-morphisms are triangle functors and a $2$-morphism between two triangle functors $F$ and $F'$ (with attached natural isomorphisms $\xi$ and $\xi'$ respectively) is a natural transformation $F \overset{\alpha}{\longrightarrow} F'$ so that the following commutes: 
\begin{center}
\begin{tikzcd}[ampersand replacement=\&]
  F\Sigma \arrow{r}{\xi} \arrow{d}{\alpha \Sigma} \& { \Sigma'F } \arrow{d}{\Sigma' \alpha} \\
F'\Sigma \arrow{r}{\xi'} \& { \Sigma'F' }
\end{tikzcd}
\end{center}

Observe that we always have $\Sigma \in \textnormal{Aut}(\mathscr{D})$ (we equip $\Sigma$ with $-1$, where $1$ is the identity of $\Sigma^2$). When obvious, we will omit the $\xi$ when speaking of triangle functors. \\
We do not require additivity in the definition of triangle and cohomological functors because it is automatic by the following proposition. 

\begin{prop}
Triangle and cohomological functors are additive.
\end{prop}

\begin{proof}
Let $\mathscr{D} \overset{F}{\longrightarrow} \mathscr{D}'$ be a triangle functor between triangulated categories (we omit the $\xi$). Considering the image of the very trivial triangle (the one with all zero vertices) of $\mathscr{D}$, we get a distinguished triangle in $\mathscr{D}'$: $$F(0) \overset{1}{\longrightarrow} F(0) \overset{1}{\longrightarrow} F(0) \longrightarrow $$ 
By \hyperref[b]{\textbf{Proposition \ref*{b}}}, $1=0$ and thus $F(0)=0$, which also tells us that the image of any zero morphism is the zero morphism. For each $X,Y \in \mathscr{D}$, since $X \longrightarrow X \oplus Y  \longrightarrow Y \overset{0}{\longrightarrow}$ is a distinguished triangle in $\mathscr{D}$, we have that $$F(X) \longrightarrow F(X \oplus Y) \longrightarrow F(Y) \overset{0}{\longrightarrow}$$ is a distinguished triangle in $\mathscr{D}'$. Since the last morphism of the latter triangle is $0$, we get by \hyperref[d]{\textbf{Proposition \ref*{d}}} that $F( X \oplus Y)= F(X) \oplus F(Y)$ in $\mathscr{D}'$, as desired. \\
The proof for cohomological functors is very similar (use the Splitting Lemma for abelian categories and the right inverse from \hyperref[d]{\textbf{Proposition \ref*{d}}}).
\end{proof}

\begin{exmp}
Recall that the right (resp. left) derived functor $$\mathscr{D}(\mathscr{A}) \overset{\mathscr{R}F}{\longrightarrow} \mathscr{D}(\mathscr{B})$$ (resp. $\mathscr{L}F$) of a left(resp. right)-exact functor $\mathscr{A} \overset{F}{\longrightarrow} \mathscr{B}$ between abelian categories, where $\mathscr{A}$ has enough injectives (resp. projectives), is the right Kan extension of $F$ along the projection of the category of chain complexes on $\mathscr{A}$ to $\mathscr{D}(\mathscr{A})$. It is easy to see that $\mathscr{R}F$ is triangulated.
\end{exmp}

\begin{prop}\label{r}
Adjoints (left or right) of triangle functors are triangle. 
\end{prop}

\begin{proof}
  Since $\mathscr{D}^{\textnormal{op}}$ is triangulated, it suffices to show the statement for, say, right adjoints. Now let $\mathscr{D} \overset{F}{\longrightarrow} \mathscr{D}'$ be a triangle functor and $\mathscr{D} \overset{G}{\longrightarrow} \mathscr{D}'$ its right adjoint. For each $X \in \mathscr{D}, Y \in \mathscr{D}'$, since $\Sigma$ is a triangle equivalence and $F$ is triangle, using adjunction property, we get: $$\textnormal{Hom}_{\mathscr{D}}(X,G(Y[1]))=\textnormal{Hom}_{\mathscr{D}}(F(X[-1]),G(Y))=\textnormal{Hom}_{\mathscr{D}}(X,G(Y)[1])$$
  By the Yoneda Lemma and arbitrariness of $X$ and $Y$, we get a natural isomorphism $\Sigma' G = G \Sigma$. Now, let $X \longrightarrow Y \longrightarrow Z \longrightarrow$ be a distinguished triangle in $\mathscr{D}'$. Taking the cone, we get a distinguished triangle $G(X) \longrightarrow G(Y) \longrightarrow  W \longrightarrow$ in $\mathscr{D}$. Using adjunction property, we get isomorphisms  $F(G(X)) \longrightarrow X$ and $F(G(Y)) \longrightarrow Y$ and we can complete them to a morphism of distinguished triangles:
\begin{center}
\begin{tikzcd}[ampersand replacement=\&]
  F(G(X)) \arrow{r} \arrow{d} \& F(G(Y)) \arrow{r} \arrow{d} \& F(W) \arrow{r} \arrow{d} \& F(G(X))[1] \arrow{d}\\
  X \arrow{r} \& Y \arrow{r} \& Z \arrow{r} \& X[1]
\end{tikzcd}
\end{center}
The morphism $F(W) \longrightarrow Z$ is an isomorphism by \hyperref[c]{\textbf{Proposition \ref*{c}}} and induces, by adjunction, an isomorphism $W \longrightarrow G(Z)$. This means that $G(X) \longrightarrow G(Y) \longrightarrow W=G(Z) \longrightarrow$ is a distingushed triangle in $\mathscr{D}$, and thus $G$ is triangle. 
\end{proof}

\begin{defn} 
An \textbf{extension-closed} subcategory of a triangulated category $\mathscr{D}$ is a full subcategory $\mathscr{C} \subseteq \mathscr{D}$ containing $0$ so that for each distinguished triangle $X \longrightarrow Y \longrightarrow Z \longrightarrow $ in $\mathscr{D}$, if $X,Z \in \mathscr{C}$, then $Y \in \mathscr{C}$. \\
We say that $\mathscr{C}$ is \textbf{thick} if it is an extension-closed triangulated subcategory. \\  
If $S \subseteq \mathscr{D}$ is a subset, we denote $\langle S \rangle$ (resp. $\langle \langle S \rangle \rangle$) the smallest extension-closed (resp. thick) subcategory of $\mathscr{D}$ containing $S$, and call it the extension-closed (resp. thick) subcategory \textbf{generated} by $S$ (we set $\langle \emptyset \rangle = 0$). 
\end{defn}

Clearly, extension-closed subcategories are additive by the remark after \hyperref[e]{\textbf{Proposition \ref*{e}}}. \\

\begin{prop}\label{f}
Let $\mathscr{D} \overset{F}{\longrightarrow} \mathscr{D}'$ be a triangle functor, $S \subseteq \mathscr{D}$ a subset. Then $F(\langle S \rangle) \subseteq \langle F(S) \rangle$ with equality if $F$ is full. A similar statement holds for cohomological functors (where the extension-closeness on the right side is thought in the abelian sense). 
\end{prop}

\begin{proof}
 We have: $$\langle S \rangle = \bigcup_{i \in \mathbb{Z}_{\ge 0}} S_i$$
 where $S_0=S$ and $S_{i+1}$ is the set of objects $X \in \mathscr{D}$ so that there is a distinguished triangle $T \longrightarrow X \longrightarrow T' \longrightarrow$ with $T, T' \in S_i$. If $X \in S_{i+1}$, applying $F$ to the latter triangle we get a distinguished triangle $F(T) \longrightarrow F(X) \longrightarrow F(T') \longrightarrow$ in $\mathscr{D}'$ and since $\langle F(S) \rangle$ is extension-closed, we conclude inductively that $F(\langle S \rangle) \subseteq \langle F(S) \rangle$. \\
 To conclude, we have to show that $F(\langle S \rangle)$ is extension-closed when $F$ is full. Pick a distinguished triangle $$F(X) \longrightarrow T \longrightarrow F(Y) \overset{f}{\longrightarrow}$$
 in $\mathscr{D}'$ with $X,Y \in \langle S \rangle$. Since $F$ is full, there is a morphism $Y[-1] \overset{g}{\longrightarrow} X$ in $\mathscr{D}$ so that $F(g)[1]=f$. Since $\langle S \rangle$ is extension-closed, $\textnormal{cone}(g) \in \langle S \rangle$, and since $F$ is triangle $F(\textnormal{cone}(g))=\textnormal{cone}(f)[-1]=T$, and thus $T \in F(\langle S \rangle)$, as desired. \\
The proof for cohomological functors is very similar. 
\end{proof}

\begin{exmp}
  The kernel (the full subcategory of objects sent to $0$) of a cohomological functor is extension-closed. The kernel of a triangle functor is thick. Indeed, using Verdier localization (see \cite{ver}) one shows that all the thick subcategories are obtained this way. 
\end{exmp}
% 
%\begin{prop}
%If $S \subseteq \mathscr{D}$ is a subset so that  $S[1]=S$, then $\langle S \rangle$ is thick.
%\end{prop}
%
%\begin{proof}
%Since $\Sigma$ is a triangle autoequivalence, by \hyperref[f]{\textbf{Proposition \ref*{f}}} and hypothesis, $\langle S \rangle [1] = \langle S \rangle$. We thus have to show that $\langle S \rangle$ is closed under cones. Let $X \longrightarrow Y \longrightarrow Z \longrightarrow $ be a distinguished triangle in $\mathscr{D}$ with $X,Y \in \langle S \rangle$. Since $X[1] \in \langle S \rangle$ by what we have already seen, rotating this triangle and applying extension-closeness we get that $Z \in \langle S \rangle$, as desired. 
%\end{proof}
%

%\newpage
%
%\subsection{Of telescopes and men}
%
%\begin{defn}\label{tel1}
%A special triangle (not necessarily distinguished) $X \overset{f}{\longrightarrow} Y \overset{g}{\longrightarrow} Z \overset{h}{\longrightarrow}$ in $\mathscr{D}$ is called \textbf{contractible} if there are morphisms $$Y \overset{u}{\longrightarrow} X$$ $$Z \overset{v}{\longrightarrow} Y$$ $$X[1] \overset{w}{\longrightarrow} Z$$ so that $uf + (hw)[-1]$, $vg + fu$ and $wh + gv$ all equal $1$. 
%\end{defn}
%
%\begin{prop}\label{tel2}
%Contractible triangles are distinguished. 
%\end{prop}
%
%\begin{proof}
%We keep notation as in \hyperref[tel1]{\textbf{Definition \ref*{tel1}}} and consider the distinguished triangle $$X \overset{f}{\longrightarrow} Y \overset{g'}{\longrightarrow} C=\textnormal{cone}(f) \overset{h'}{\longrightarrow}$$ 
%Applying the cohomological functor $\textnormal{Hom}_{\mathscr{D}}(Z,*)$ to this triangle and reasoning as in the proof of \hyperref[g]{\textbf{Proposition \ref*{g}}} we get a morphism $Z \overset{\phi}{\longrightarrow} C$ so that $h' \phi = h$. Define $$c=g'v + \phi w h$$
%The following is then a morphism of special triangles by an easy check: 
%\begin{center}
%\begin{tikzcd}[ampersand replacement=\&]
%X \arrow{r} \arrow{d}{1} \& Y \arrow{r} \arrow{d}{1} \& Z \arrow{d}{c} \arrow{r} \& X[1] \arrow{d}{1} \\
%X \arrow{r} \& Y \arrow{r} \& C \arrow{r} \& X[1]
%\end{tikzcd}
%\end{center}
%Using \hyperref[c]{\textbf{Proposition \ref*{c}}} we see that $c$ is an isomorphism, as desired. 
%\end{proof}
%
%\begin{defn}
%Let 
%\begin{center}
%\begin{tikzcd}[ampersand replacement=\&]
%X \arrow{r}{f} \arrow{d}{a} \& Y \arrow{r}{g} \arrow{d}{b} \& Z \arrow{d}{c} \arrow{r}{h} \& X[1] \arrow{d}{a[1]} \\
%X' \arrow{r}{f'} \& Y' \arrow{r}{g'} \& Z' \arrow{r}{h'} \& X'[1]
%\end{tikzcd}
%\end{center}
%be a morphism of (not necessarily distinguished) triangles in $\mathscr{D}$. We call its \textbf{telescope} the (not necessarily distinguished) triangle 
%$$Y \oplus X' \xrightarrow{\begin{pmatrix}  -g & 0 \\ b & f'  \end{pmatrix}} Z \oplus Y' \xrightarrow{\begin{pmatrix}  -h & 0 \\ c & g' \end{pmatrix}} X[1]\oplus Z' \xrightarrow{\begin{pmatrix}  -f[1] & 0 \\ a[1] & h'  \end{pmatrix}} $$
%\end{defn}
%
%\begin{prop}
%In hypothesis (1) and notation of \hyperref[a]{\textbf{Proposition \ref*{a}}}, the completed morphism can be taken so that its telescope is distinguished. 
%\end{prop}
%
%\begin{proof}
%First of all, complete $a$ and $b$ to a morphism of distinguished triangles. Consider the triangles $$X' \oplus Y \xrightarrow{\begin{pmatrix}  1 & 0 \\ 0 & 1 \\ f' & b  \end{pmatrix}} X' \oplus Y \oplus Y' \xrightarrow{(-f' \ -b \ 1)} Y' \overset{0}{\longrightarrow}$$ 
%$$X' \oplus X \xrightarrow{\begin{pmatrix}  1 & 0 \\ 0 & -f \\ 0 & 0  \end{pmatrix}} X' \oplus Y \oplus Y' \xrightarrow{\begin{pmatrix}  0 & 0 & 1 \\ 0 & -g & 0  \end{pmatrix}} Y' \oplus Z \xrightarrow{\begin{pmatrix}  0 & 0 \\ 0 & -h  \end{pmatrix}}$$
%The first one is contractible and thus distinguished by \hyperref[tel2]{\textbf{Proposition \ref*{tel2}}}. The second one, being the direct sum of $X \longrightarrow Y \longrightarrow Z \longrightarrow$ and the contractible $X' \longrightarrow X' \oplus Y' \longrightarrow Y' \longrightarrow $, is distinguished by \hyperref[e]{\textbf{Proposition \ref*{e}}}. 
%\end{proof}
%
%\begin{defn}
%A commutative square in $\mathscr{D}$ 
%\begin{center}
%\begin{tikzcd}[ampersand replacement=\&]
%X \arrow{r}{f'} \arrow{d}{g'} \& Y' \arrow{d}{g} \\
%Y \arrow{r}{f} \& Z 
%\end{tikzcd}
%\end{center}
%is called \textbf{homotopy cartesian} if there is a distinguished triangle $$X \overset{\binom{g'}{-f'}}{\longrightarrow} Y \oplus Y' \overset{(f \ g)}{\longrightarrow} Z \longrightarrow $$ 
%The last map of the above triangle is called \textbf{differential} of the square. 
%\end{defn}
%
%\begin{prop}
%Let $X \overset{f}{\longrightarrow} Y \longrightarrow Z \longrightarrow$ and $X \longrightarrow Y' \longrightarrow Z' \overset{h}{\longrightarrow}$ distinguished triangles of $\mathscr{D}$, $Y \overset{b}{\longrightarrow} Y'$ a morphism. If the first square of the below diagram commutes, then $b$ extends to a morphism of triangles
%\begin{center}
%\begin{tikzcd}[ampersand replacement=\&]
%X \arrow{r}{f} \arrow{d}{1} \& Y \arrow{r} \arrow{d}{b} \& Z \arrow{d}{c} \arrow{r} \& X[1] \arrow{d}{1} \\
%X \arrow{r} \& Y' \arrow{r} \& Z' \arrow{r}{h} \& X[1]
%\end{tikzcd}
%\end{center}
%so that the central square is homotopy cartesian with differential $f[1]h$. 
%\end{prop}
%
\newpage 

\subsection{Hearts and towers} 
As opposed to derived categories, triangulated categories do not a priori contain any canonical abelian category. Despire the appearence, this is an advantage for us as we will in future consider many of such subcategories (called 'hearts'), and noone of them will be privileged. We want thus to put extra structure on our triangulated categories in order to recover the behavior of an abelian category inside its derived one. This will be done throught the formalism of 't-structures', first introduced in \cite{del} and deeply developed in \cite{gel}. \\
From now on, for set-theoretic purposes, we assume that $\mathscr{D}$ is (essentially) small. \\

\begin{defn}
A \textbf{t-structure} on $\mathscr{D}$ is a full additive subcategory $\mathfrak{t} \subseteq \mathscr{D}$ so that: 
\begin{enumerate}
\item $\mathfrak{t}[1] \subseteq \mathfrak{t}$
\item For each $X \in \mathscr{D}$ there is a distinguished triangle $$T \longrightarrow X \longrightarrow T' \longrightarrow $$ so that $T \in \mathfrak{t}$ and $T' \in \mathfrak{t}^{\perp}$\footnote{For a subset $S \subseteq \mathscr{D}$, $S^{\perp}$ denotes the full (extension-closed) subcategory consisting of $X \in \mathscr{D}$ so that $\textnormal{Hom}_{\mathscr{D}}(S,X)=0$.}
\end{enumerate} 
We denote $\mathfrak{ts}(\mathscr{D})$ the poset of t-structures on $\mathscr{D}$ (ordered by opposite inclusion). 
\end{defn}

Observe that $\mathfrak{ts}(\mathscr{D})$ is a bounded poset: the maximum is the zero subcategory, the minimum the whole $\mathscr{D}$. \\

\begin{prop}
Let $\mathfrak{t}$ be a t-structure on $\mathscr{D}$. Associating to $X \in \mathscr{D}$ the triangle of (2) defines a functor (unique up to isomorphism) $$\mathscr{D} \overset{\tau_{\mathfrak{t}}}{\longrightarrow} \bigtriangleup$$ Moreover, composing $\tau_{\mathfrak{t}}$ with the projection to the left (resp. right) vertex defines a right (resp. left) adjoint to the inclusion $\mathfrak{t} \subseteq \mathscr{D}$ (resp. $\mathfrak{t}^{\perp} \subseteq \mathscr{D}$) which we denote $\tau^{\ge}_{\mathfrak{t}}$ (resp. $\tau^{<}_{\mathfrak{t}}$). 
\end{prop}

\begin{proof}
Let $X \longrightarrow Y$ be a morphism in $\mathscr{D}$. Using \textit{(2)} we have distinguished triangles $T \longrightarrow X \longrightarrow T' \longrightarrow $ and  $F \longrightarrow Y \longrightarrow F' \longrightarrow $. Since $T[1] \in \mathfrak{t}$ by \textit{(1)} and $F' \in \mathfrak{t}^{\perp}$, we have $\textnormal{Hom}_{\mathscr{D}}(T[1],F')=\textnormal{Hom}_{\mathscr{D}}(T,F'[-1])=0$ and thus by \hyperref[g]{\textbf{Proposition \ref*{g}}} we get a unique, functorial morphism between our two triangles. This shows the first part of the statement. Thus, we denote $\tau_{\mathfrak{t}}(X)$ the triangle of $X$, $\tau^{\ge}_{\mathfrak{t}}(X)=T$ and $\tau^{<}_{\mathfrak{t}}(X)=T'$. \\
Now, pick $Z \in \mathfrak{t}$. Since again $\textnormal{Hom}_{\mathscr{D}}(Z,T')=\textnormal{Hom}_{\mathscr{D}}(Z,T'[-1])=0$, applying the cohomological functor $\textnormal{Hom}_{\mathscr{D}}(Z,*)$ to the distinguished triangle $$\tau^{\ge}_{\mathfrak{t}}(X) \longrightarrow X \longrightarrow \tau^{<}_{\mathfrak{t}}(X) \longrightarrow $$ we get $$\textnormal{Hom}_{\mathscr{D}}(Z,\tau^{\ge}_{\mathfrak{t}}(X))=\textnormal{Hom}_{\mathscr{D}}(Z,X)$$ which is the desired adjunction property. Similarly, picking $Z \in \mathfrak{t}^{\perp}$ and applying $\textnormal{Hom}_{\mathscr{D}}(*,Z)$ to the same triangle, we get  $$\textnormal{Hom}_{\mathscr{D}}(\tau^{<}_{\mathfrak{t}}(X),Z)=\textnormal{Hom}_{\mathscr{D}}(X,Z)$$ which concludes the proof. 
\end{proof}

Since adjoints of additive functors are additive, $\tau^{\ge}_{\mathfrak{t}}$ and $\tau^{<}_{\mathfrak{t}}$ are additive and thus $\tau_{\mathfrak{t}}$ is additive too (this also follows from \hyperref[e]{\textbf{Proposition \ref*{e}}}).
Now, besides $\textnormal{ts}(\mathscr{D})$ is not functorial in $\mathscr{D}$, it is still well-defined: taking the image $F(\mathfrak{t})$ of a t-structure $\mathfrak{t}$ under a triangle autoequivalense $F$ defines an action of $\textnormal{Aut}(\mathscr{D})$ on $\textnormal{ts}(\mathscr{D})$, and we have: $$\tau_{F( \mathfrak{t})} = F \tau_{\mathfrak{t}} F^{-1}$$

\begin{prop}\label{h}
Associanting to a t-structure $\mathfrak{t}$ the functor $\tau_{\mathfrak{t}}$ defines a functor $$\mathfrak{ts}(\mathscr{D})^{\textnormal{op}} \overset{\tau_*}{\longrightarrow} \bigtriangleup^{\mathscr{D}}$$
($\mathfrak{ts}(\mathscr{D})$ is thought as a posetal category).
\end{prop}

\begin{proof}
Let $\mathfrak{t} \subseteq \mathfrak{q}$ be t-structures on $\mathscr{D}$, $X \in \mathscr{D}$. Since $\tau^{\ge}_{\mathfrak{t}}(X) \in \mathfrak{t}$ and $\tau^{<}_{\mathfrak{q}}(X) \in \mathfrak{q}^{\perp} \subseteq \mathfrak{t}^{\perp}$, we can again use \hyperref[g]{\textbf{Proposition \ref*{g}}} to extend the identity of $X$ to a morphism of triangles $\tau_{\mathfrak{t}}(X) \longrightarrow \tau_{\mathfrak{q}}(X)$. This is the desired functorial natural transformation. 
\end{proof}

\begin{exmp}\label{bbt}
Let $\mathscr{A}$ be an abelian category. Denote by $\mathfrak{t}$ the full subcategory of complexes with cohomology concentrated in strictly negative degree. Then $\mathfrak{t}$ is a t-structure on $\mathscr{D}(\mathscr{A})$ called standard t-structure, and the functors $\tau_{\mathfrak{t}}^{\ge}$ and $\tau_{\mathfrak{t}}^{<}$ correspond to the truncation of a complex on the right and on the left of $0$ respectively. Indeed, the the name 't-structure' stands for 'truncation structure'. 
\end{exmp}

\begin{prop}\label{m}
Let $\mathfrak{t}$ be a t-structure on $\mathscr{D}$, $X \longrightarrow Y \longrightarrow Z \longrightarrow $ a distingusihed triangle in $\mathscr{D}$. If $Z \in \mathfrak{t}^{\perp}$, then $\tau^{\ge}_{\mathfrak{t}}(X)=\tau^{\ge}_{\mathfrak{t}}(Y)$. If $X \in \mathfrak{t}$, then $\tau^{<}_{\mathfrak{t}}(Y)=\tau^{<}_{\mathfrak{t}}(Z)$.
\end{prop}

\begin{proof}
Pick $W \in \mathfrak{t}$. Applying the cohomological functor $\textnormal{Hom}_{\mathscr{D}}(W,*)$ to the triangle and using adjunction property, we see $$\textnormal{Hom}_{\mathscr{D}}(W,\tau^{\ge}_{\mathfrak{t}}(X))=\textnormal{Hom}_{\mathscr{D}}(W,\tau^{\ge}_{\mathfrak{t}}(Z))$$
By arbitrariness of $W$, using the Yoneda Lemma (applied to the category $\mathfrak{t}$), we get the result. \\
The proof of the second statement is very similar (just pick $W \in  \mathfrak{t}^{\perp}$ and apply $\textnormal{Hom}_{\mathscr{D}}(*,W)$ instead).
\end{proof}

\begin{prop}
Let $\mathfrak{t}$ be a t-structure on $\mathscr{D}$. Then $\mathfrak{t}$ and $\mathfrak{t}^{\perp}$ are extension-closed. 
\end{prop}

\begin{proof}
Let $X \longrightarrow Y \longrightarrow Z \longrightarrow$ be a distinguished triangle with $X,Z \in \mathfrak{t}$. Then $\tau^<_{\mathfrak{t}}(Y)=\tau^<_{\mathfrak{t}}(Z)=0$ by \hyperref[m]{\textbf{Proposition \ref*{m}}}, which means that $Y \in \mathfrak{t}$, as desired. The proof for $\mathfrak{t}^{\perp}$ is very similar. 
\end{proof}

\begin{prop}\label{l}
If $\mathfrak{t} \subseteq \mathfrak{q}$ are t-structures on $\mathscr{D}$, then: $$\tau^{\ge}_{\mathfrak{t}}\tau^<_{\mathfrak{q}}=\tau^<_{\mathfrak{q}}\tau^{\ge}_{\mathfrak{t}}=0$$ $$\tau^{\ge}_{\mathfrak{t}}\tau^{\ge}_{\mathfrak{q}}=\tau^{\ge}_{\mathfrak{q}}\tau^{\ge}_{\mathfrak{t}}=\tau^{\ge}_{\mathfrak{t}}$$ $$\tau^<_{\mathfrak{q}}\tau^<_{\mathfrak{t}}=\tau^<_{\mathfrak{t}}\tau^<_{\mathfrak{q}}=\tau^<_{\mathfrak{q}}$$ $$\tau^<_{\mathfrak{t}}\tau^{\ge}_{\mathfrak{q}}=\tau^{\ge}_{\mathfrak{q}}\tau^<_{\mathfrak{t}}$$
Moreover, the last functor above coincides pointwise with the cone of  the natural transformation $\tau^{\ge}_{\mathfrak{t}} \longrightarrow \tau^{\ge}_{\mathfrak{q}}$ induced by the inclusion $\mathfrak{t} \subseteq \mathfrak{q}$.
\end{prop}

\begin{proof}
The first identity follows from the definitions. Now, pick $X \in \mathscr{D}$. By applying the \hyperref[s]{\textbf{$3 \times 3$ Lemma}} to the morphism of triangles $\tau_{\mathfrak{t}}(X) \longrightarrow \tau_{\mathfrak{q}}(X)$ induced by the inclusion $\mathfrak{t} \subseteq \mathfrak{q}$ (\hyperref[h]{\textbf{Proposition \ref*{h}}}), we obtain a commutative diagram: 

\begin{center}
\begin{tikzcd}[ampersand replacement=\&]
\tau^{\ge}_{\mathfrak{t}}(X) \arrow{r} \arrow{d} \& X \arrow{r} \arrow{d}{1} \& \tau^<_{\mathfrak{t}}(X) \arrow{r} \arrow{d} \& \tau^{\ge}_{\mathfrak{t}}(X) [1] \arrow{d} \\
\tau^{\ge}_{\mathfrak{q}}(X) \arrow{r} \arrow{d} \& X \arrow{r} \arrow{d} \& \tau^<_{\mathfrak{q}}(X) \arrow{r} \arrow{d} \& \tau^{\ge}_{\mathfrak{q}}(X)[1] \arrow{d} \\
A \arrow{r} \arrow{d} \& 0 \arrow{r} \arrow{d} \& B \arrow{r} \arrow{d} \& A[1] \arrow{d}  \\
\tau^{\ge}_{\mathfrak{t}}(X)[1] \arrow{r}  \&  X[1] \arrow{r} \& \tau^<_{\mathfrak{t}}(X)[1] \arrow{r} \&  \tau^{\ge}_{\mathfrak{t}}(X)[2]
\end{tikzcd}
\end{center}
Since $\tau^{\ge}_{\mathfrak{t}}(X) \in \mathfrak{t} \subseteq \mathfrak{q}$, by rotating the left column and applying extension closeness we get that $A[-1] \in \mathfrak{q}$ and thus, since $\mathfrak{q}[1] \subseteq \mathfrak{q}$, $A \in \mathfrak{q}$. Similarly, $B \in \mathfrak{t}^{\perp}$. Since $B=A[1]$ by \hyperref[i]{\textbf{Proposition \ref*{i}}}, we conclude $A \in \mathfrak{t}^{\perp}$ and $B \in \mathfrak{q}$. But then the left column is just $\tau_{\mathfrak{t}}(\tau^{\ge}_{\mathfrak{q}}(X))$ and the third column is just the rotation of $\tau_{\mathfrak{q}}(\tau^<_{\mathfrak{t}}(X))$, as desired. 
\end{proof}

Let $\mathfrak{t}$ be a t-structure on $\mathscr{D}$.  \\

\begin{defn}
We call \textbf{heart} of $\mathfrak{t}$ the full additive subcategory $$\heartsuit_{\mathfrak{t}}=\mathfrak{t}\cap \mathfrak{t}[1]^{\perp}$$
We call \textbf{cohomology} in grade $n \in \mathbb{Z}$ induced by $\mathfrak{t}$ the additive functor $$H^n_{\mathfrak{t}}=\tau^{\ge}_{\mathfrak{t}[n]} \tau^<_{\mathfrak{t}[n+1]}=\tau^<_{\mathfrak{t}[n+1]}\tau^{\ge}_{\mathfrak{t}[n]}$$ 
which takes values in $\heartsuit_{\mathfrak{t}}[n]$. \\
%We call \textbf{support} of $X \in \mathscr{D}$ with respect to $\mathfrak{t}$ the subset of $\mathbb{Z}$ $$\textnormal{supp}_{\mathfrak{t}}(X)=\{ n \in \mathbb{Z} \ | \ H^n_{\mathfrak{t}}(X) \not = 0 \}$$ 
\end{defn}
%
%\begin{prop}  
%Let $X \longrightarrow Y \longrightarrow Z \longrightarrow $ be a distinguised triangle in $\mathscr{D}$. If $X,Y \in \heartsuit_{\mathfrak{t}}$, then there is a distinguished triangle $$H^1_{\mathfrak{t}}(Z) \longrightarrow Z \longrightarrow H^0_{\mathfrak{t}}(Z) \longrightarrow$$
%\end{prop}
%
%\begin{proof} 
%Consider the rotated triangle $$Y \longrightarrow Z \longrightarrow X[1] \longrightarrow$$
%Since $Y$ and $X[1]$ are semistable of phase $0$ and $1$ and  $\mathscr{P}^{\heartsuit}_{[0,1]}$ is extension-closed, we have $0 \le \phi_{\heartsuit}^{\pm}(Z) \le 1$ and thus $\phi_{\heartsuit}^{\pm}(Z) \in \{0,1 \}$, which is what we wanted.
%\end{proof}
%
%The above proposition simply means that, if we denote $f$ the first morphism of the triangle, the Postnikov tower of $Z$ with respect to $\heartsuit$ is simply $$0 \longrightarrow H^1_{\heartsuit}(Z) \longrightarrow Z$$ and we have a distinguished triangle 
Clearly, $X \in \heartsuit_{\mathfrak{t}}[n]$ if and only if $H^n_{\mathfrak{t}}(X)=X$. \\

\begin{prop}\label{o}
$\heartsuit_{\mathfrak{t}}$ is an abelian category. Moreover, a short exact sequence $0 \longrightarrow X \longrightarrow Y \longrightarrow Z \longrightarrow 0$ in $\heartsuit_{\mathfrak{t}}$ induces a distinguished triangle $X \longrightarrow Y \longrightarrow Z \longrightarrow $ in $\mathscr{D}$. 
\end{prop}

\begin{proof}
Let $X \overset{f}{\longrightarrow} Y$ be a morphism in $\heartsuit_{\mathfrak{t}}$. We define: $$\textnormal{ker}(f)=H^0_{\mathfrak{t}}(\textnormal{cone}(f)[-1])  \overset{p}{\longrightarrow} X $$  $$Y \overset{q}{\longrightarrow} \textnormal{coker}(f)=H^0_{\mathfrak{t}}(\textnormal{cone}(f))$$
where the maps $p,q$ are obtained by applying the functor $H^0_{\mathfrak{t}}$ to the triangle (and to its rotation) $X \longrightarrow Y \longrightarrow \textnormal{cone}(f)  \longrightarrow$ and using $H^0_{\mathfrak{t}}(X)=X$ and $H^0_{\mathfrak{t}}(Y)=Y$. \\
Now, rotating the latter triangle and using extension-closeness, we see that $\textnormal{cone}(f) \in \mathfrak{t} \cap \mathfrak{t}[2]^{\perp}$, and thus $$\textnormal{ker}(f)=\tau^{\ge}_{\mathfrak{t}[1]}(\textnormal{cone}(f))[-1]$$ $$\textnormal{coker}(f)=\tau^{<}_{\mathfrak{t}[1]}(\textnormal{cone}(f))$$
This means that $\tau_{\mathfrak{t}[1]}(\textnormal{cone}(f))$ is just $$\textnormal{ker}(f)[1] \longrightarrow \textnormal{cone}(f) \longrightarrow \textnormal{coker}(f) \longrightarrow $$
Since $q$ is the composition $Y \longrightarrow \textnormal{cone}(f) \longrightarrow \textnormal{coker}(f)$, by taking the generated braid we get a distinguished triangle $$X[1] \longrightarrow \textnormal{cone}(q) \longrightarrow  \textnormal{ker}(f)[2] \overset{p[2]}{\longrightarrow} $$
We have shown that $\textnormal{cone}(q) = \textnormal{cone}(p)[1]$, and thus $\textnormal{im}(f)=\textnormal{ker}(q)=\textnormal{coker}(p)=\textnormal{coim}(f)$. Images and coimages coincide: this shows that $\heartsuit_{\mathfrak{t}}$ is abelian. \\
Now if $\textnormal{ker}(f)=0$, by looking at  $\tau_{\mathfrak{t}[1]}(\textnormal{cone}(f))$ above we see that $\textnormal{cone}(f)=\textnormal{coker}(f)$, which means that the cokernel of a monomorphisms in $\heartsuit_{\mathfrak{t}}$ is the cone of that same morphism in $\mathscr{D}$, which is a reformulation of the second part of the statement.
 
\end{proof}

Observe that conversely, if $X \longrightarrow Y \longrightarrow Z \longrightarrow$ is a distinguished triangle in $\mathscr{D}$ with $X,Y,Z \in \heartsuit_{\mathfrak{t}}$, then $H^1_{\mathfrak{t}}(Z)=0$ and $H^0_{\mathfrak{t}}(Z)=Z$, and thus $0 \longrightarrow X \longrightarrow Y \longrightarrow Z \longrightarrow 0$ is an exact sequence in $\heartsuit_{\mathfrak{t}}$. This also implies that $\heartsuit_{\mathfrak{t}}[n]$ is a heart for each $n \in \mathbb{Z}$, and $\Sigma^n$ restricted to $\heartsuit_{\mathfrak{t}}$ is an exact equivalence of abelian categories. \\

\begin{prop}
For each $X,Y \in \heartsuit_{\mathfrak{t}}$ there is a (canonical) isomorphism of groups:  $$\textnormal{Ext}^1_{\heartsuit_{\mathfrak{t}}}(X,Y)=\textnormal{Hom}_{\mathscr{D}}(X,Y[1])$$ 
\end{prop}

\begin{proof}
If $0 \longrightarrow Y \longrightarrow Z \longrightarrow X \longrightarrow 0$ is an extension, by the second claim of \hyperref[o]{\textbf{Proposition \ref*{o}}} we get a distinguished triangle $Y \longrightarrow Z \longrightarrow X \longrightarrow$ the last arrow of which is an element of $\textnormal{Hom}_{\mathscr{D}}(X,Y[1])$. Conversely, if we have a morphism $X \longrightarrow Y[1]$, by taking its cone $Z$ we see by extension-closeness that $Z[-1] \in \heartsuit_{\mathfrak{t}}$ and is thus an extension by the above observation.  
\end{proof}

\begin{exmp}\label{lla}
If $\mathscr{A}$ is an abelian category with enough injectives, then something more is true by an easy check: for each $X,Y \in \mathscr{A}$, $n \ge 0$, $$\textnormal{Ext}_{\mathscr{A}}^n(X,Y)=\textnormal{Hom}_{\mathscr{D}(\mathscr{A})}(X,Y[n])$$
where, as usual, we denote $\textnormal{Ext}_{\mathscr{A}}^i(*,Y)=H^i\mathscr{R}\textnormal{Hom}_{\mathscr{A}}(*,Y)$.
\end{exmp}

\begin{prop}
For each $n \in \mathbb{Z}$, the functor $\mathscr{D} \overset{H^n_{\mathfrak{t}}}{\longrightarrow} \heartsuit_{\mathfrak{t}}[n]$  is cohomological. 
\end{prop}

\begin{proof}
Since $H^n_{\mathfrak{t}}=H^0_{\mathfrak{t}[n]}$, we can assume $n=0$. Pick a distinguished triangle $X \longrightarrow Y \longrightarrow Z \longrightarrow$ in $\mathscr{D}$. \\
First, suppose that $X,Z \in \mathfrak{t}[1]^{\perp}$, and hence $Y \in \mathfrak{t}[1]^{\perp}$ too by extension-closeness. This means that $H^0_{\mathfrak{t}}$ and $\tau^{\ge}_{\mathfrak{t}}$ coincide on $X,Y,Z$. Now, for $W \in \heartsuit_{\mathfrak{t}}$, applying the cohomological functor $\textnormal{Hom}_{\mathscr{D}}(W,*)$ to the triangle and using adjunction property we get an exact sequence of abelian groups: $$\textnormal{Hom}_{\mathscr{D}}(W,H_{\mathfrak{t}}^0(X)) \longrightarrow \textnormal{Hom}_{\mathscr{D}}(W,H_{\mathfrak{t}}^0(Y)) \longrightarrow \textnormal{Hom}_{\mathscr{D}}(W,H_{\mathfrak{t}}^0(Z))$$
By arbitrariness of $W$ and since the Yoneda embedding of an abelian category reflects exactness, we get that $$H_{\mathfrak{t}}^0(X) \longrightarrow H_{\mathfrak{t}}^0(Y) \longrightarrow H_{\mathfrak{t}}^0(Z)$$ is exact in $\heartsuit_{\mathfrak{t}}$. \\
Now, suppose just $Z \in \mathfrak{t}[1]^{\perp}$. By taking the cone of the composition $\tau^{\ge }_{\mathfrak[1]}(X) \longrightarrow X \longrightarrow Y$ we get a distinguished triangle $$\tau^{\ge }_{\mathfrak[1]}(X) \longrightarrow Y \longrightarrow T \longrightarrow $$ and we have $\tau^<_{\mathfrak[1]}(Y)=\tau^<_{\mathfrak[1]}(T)$ by \hyperref[m]{\textbf{Proposition \ref*{m}}}, and hence $H^0_{\mathfrak{t}}(Y)=H^0_{\mathfrak{t}}(T)$. By taking the braid generated by the composition $\tau^{\ge}_{\mathfrak{t}[1]}(X) \longrightarrow X \longrightarrow Y$, we get a distinguished triangle $\tau^{<}_{\mathfrak{t}[1]}(X) \longrightarrow T \longrightarrow Z \longrightarrow $, and applying $H^0_{\mathfrak{t}}$, since we are in the situation above and $H^0_{\mathfrak{t}}(\tau^{<}_{\mathfrak{t}[1]}(X))=H^0_{\mathfrak{t}}(X)$, we get the result. \\
Now, in general, taking the braid generated by the composition $Y \longrightarrow Z  \longrightarrow \tau^{<}_{\mathfrak{t}[1]}(Z)$ we get a distinguished triangle $$X[1] \longrightarrow T' \longrightarrow \tau^{\ge }_{\mathfrak{t}[2]}(Z[1]) \longrightarrow $$
Again by \hyperref[m]{\textbf{Proposition \ref*{m}}} $\tau^{<}_{\mathfrak{t}[2]}(X[1])=\tau^{<}_{\mathfrak{t}[2]}(T')$ and hence $H^1_{\mathfrak{t}}(X[1])=H^1_{\mathfrak{t}}(T')$. Applying then $H^0_{\mathfrak{t}}$ to the triangle $T'[-1] \longrightarrow Y \longrightarrow \tau^{<}_{\mathfrak{t}[1]}(Z) \longrightarrow $, since we are in the situation above and $H^0_{\mathfrak{t}}(\tau^{<}_{\mathfrak{t}[1]}(Z))=H^0_{\mathfrak{t}}(Z)$, we get the result. 
\end{proof}

\begin{defn}
We say that $\mathfrak{t}$ is \textbf{bounded} if $$\mathscr{D}=\bigcup_{n,m \in \mathbb{Z}} \mathfrak{t}[n] \cap \mathfrak{t}[m]^{\perp}$$
We denote $\mathfrak{bts}(\mathscr{D}) \subseteq \mathfrak{ts}(\mathscr{D})$ the set of bounded t-structures on $\mathscr{D}$. 
\end{defn}

\begin{exmp}
Let $\mathscr{A}$ be an abelian category and consider the standard t-structure $\mathfrak{t}$ on $\mathscr{D}(\mathscr{A})$ from \hyperref[bbt]{\textbf{Example \ref*{bbt}}}. Then $\heartsuit_{\mathfrak{t}}=\mathscr{A}$, with equivalence given by sending objects of $\mathscr{A}$ to the corresponding complexes concentrated in degree $0$. Moreover, we have $H^i=H^{-i}_{\mathfrak{t}}$ for each $i \in \mathbb{Z}$. While $\mathfrak{t}$ is not bounded in $\mathscr{D}(\mathscr{A})$, its intersection is bounded in $\mathscr{D}^b(\mathscr{A})$.
\end{exmp}

Observe that if $\mathfrak{t}$ is bounded, then for each $X \in \mathscr{D}$ the set of indices $n \in \mathbb{Z}$ in which its cohomology doesn't vanish is finite. 

\begin{prop}\label{p}
Assume that $\mathfrak{t}$ is bounded. Then for each $0 \not = X \in \mathscr{D}$ there is a finite set $\{k_1 > \cdots > k_n \} \subseteq \mathbb{Z}$ and a factorization of the initial morphism, called \textbf{Postnikov tower} of $X$ with respect to $\mathfrak{t}$, $$0=X_0 \overset{\alpha_1}{\longrightarrow} \cdots \overset{\alpha_n}{\longrightarrow} X_n=X$$ 
so that $\textnormal{cone}(\alpha_i)=H^{k_i}_{\mathfrak{t}}(X)$ for each $1 \le i \le n$. 
\end{prop}

\begin{proof}
Take as the set the indices in which the cohomology of $X$ doesn't vanish and put $X_i=\tau^{\ge}_{\mathfrak{t}[k_i]}(X)$. Now, since $k_i > k_{i+1}$, we have an induced natural transformation (valuated in $X$) $\tau^{\ge}_{\mathfrak{t}[k_i]}(X) \overset{\alpha_i}{\longrightarrow} \tau^{\ge}_{\mathfrak{t}[k_{i+1}]}(X)$. Now, by the last claim of \hyperref[h]{\textbf{Proposition \ref*{h}}} we have $$\textnormal{cone}(\alpha_i)=\tau^<_{\mathfrak{t}[k_i]}(\tau^{\ge}_{\mathfrak{t}[k_{i+1}]}(X))$$
But for each $k_{i+1}< k < k_i$, since $H^k_{\mathfrak{t}}(X)=0$ we see inductively by the relations in \hyperref[h]{\textbf{Proposition \ref*{h}}} that $\tau^{\ge}_{\mathfrak{t}[k]}(X)=\tau^{\ge}_{\mathfrak{t}[k_i+1]}(X)$ and thus $\textnormal{cone}(\alpha_i)=H^{k_i}_{\mathfrak{t}}(X)$, as desired. 
\end{proof}

%\begin{defn}
 %The \textbf{homological dimension} of $\mathfrak{t}$ is the least integer (if it exists) $n \in \mathbb{Z}_{\ge 0}$ so that for each $X,Y \in \heartsuit_{\mathfrak{t}}$ and $0 \le i \le n$ $$\textnormal{Hom}_{\mathscr{D}}(X,Y[i])=0$$
%\end{defn}
%
%\begin{prop}
%If $\mathfrak{t}$ is bounded and has homological dimension $\le 1$, then for each $X \in \mathscr{D}$ $$X=\bigoplus_{n \in \mathbb{Z}} H^{n}_{\mathfrak{t}}(X)$$
%\end{prop}
%
%\begin{proof}
%By looking at the last step of the Postnikov tower of $X$ with respect to $\mathfrak{t}$, we have a distinguished triangle $$T \longrightarrow X \longrightarrow H \longrightarrow $$
%By applying the cohomological functor $\textnormal{Hom}_{\mathscr{D}}(H,*)$ to the Postnikov tower of $T[1]$, we see that $\textnormal{Hom}_{\mathscr{D}}(H,T[1])=0$ and thus the last morphism of the above triangle is $0$. By \hyperref[d]{\textbf{Proposition \ref*{d}}} $X=H \oplus T$, and we can conclude by induction on the length of the Postnikov tower. 
%\end{proof}
%
%Now, is it now clear if $\mathfrak{ts}(\mathscr{D})$ is a lattice. It is indeed not always true that the intersection of two t-structures is again a t-structure nor that the extension-closure of the union of two t-structures is a t-structure. We shall give sufficient conditions for that. \\
%
%\begin{defn}
%An ordered pair $(\mathfrak{t},\mathfrak{q})$ of t-structures is said to be \textbf{lower} (resp. \textbf{upper}) \textbf{consistent} if $$\tau_{\mathfrak{t}}^{\ge }(\mathfrak{q}) \subseteq \mathfrak{q}$$
%(resp. $\tau_{\mathfrak{q}}^<(\mathfrak{t}^{\perp}) \subseteq \mathfrak{t}^{\perp}$). We call $(\mathfrak{t},\mathfrak{q})$ \textbf{consistent} if it is both upper and lower consistent. 
%\end{defn}
%
%Observe that if $(\mathfrak{t},\mathfrak{q})$ is lower consistent, then rotating the triangles $\tau_{\mathfrak{t}}(\mathfrak{q})$ and using extension-closeness of $\mathfrak{q}$ we see that $$\tau_{\mathfrak{t}}^<(\mathfrak{q}) \subseteq \mathfrak{q}$$
%and similarly for upper consistency. \\
%
%\begin{prop}\label{baa}
%If $(\mathfrak{t},\mathfrak{q})$ is lower (resp. upper) consistent, then $\mathfrak{t} \cap \mathfrak{q}$ is a t-structure on $\mathscr{D}$ with $$\tau_{\mathfrak{t} \cap \mathfrak{q}}^{\ge}=\tau_{\mathfrak{t}}^{\ge} \tau_{\mathfrak{q}}^{\ge}$$
%(resp. $\mathfrak{t}^{\perp} \cap \mathfrak{q}^{\perp}$ is a t-structure with $\tau_{\mathfrak{t}^{\perp} \cap \mathfrak{q}^{\perp}}^<=\tau_{\mathfrak{q}}^< \tau_{\mathfrak{t}}^<$).
%Moreover, if both $\mathfrak{t}$ and $\mathfrak{q}$ are bounded then so is $\mathfrak{t} \cap \mathfrak{q}$ (resp. $\mathfrak{t}^{\perp} \cap \mathfrak{q}^{\perp}$).
%\end{prop}
%
%\begin{proof}
%We'll just prove the claim in the lower setting (the upper one is obtained by passing to the opposite category). Clearly, $(\mathfrak{t} \cap \mathfrak{q})[1]=\mathfrak{t}[1] \cap \mathfrak{q}[1] \subseteq \mathfrak{t} \cap \mathfrak{q}$. Now pick $X \in \mathscr{D}$. By definition of t-structure, we have morphisms: $$\tau_{\mathfrak{t}}^{\ge} (\tau_{\mathfrak{q}}^{\ge}(X)) \longrightarrow  \tau_{\mathfrak{q}}^{\ge}(X) \longrightarrow X$$
%Now, $\tau_{\mathfrak{t}}^{\ge} \tau_{\mathfrak{q}}^{\ge}(X) \in \mathfrak{t} \cap \mathfrak{q}$ by consistency. Denote $T$ the cone of the above composition. By braiding the latter we get a distinguished triangle $$\tau_{\mathfrak{t}}^<(\tau_{\mathfrak{q}}^{\ge}(X)) \longrightarrow T \longrightarrow \tau_{\mathfrak{q}}^<(X) \longrightarrow$$ 
%Since $(\mathfrak{t} \cap \mathfrak{q})^{\perp}$ is extension-closed, we see that $T \in (\mathfrak{t} \cap \mathfrak{q})^{\perp}$, and this shows that $\mathfrak{t} \cap \mathfrak{q}$ is a t-structure. \\
%Suppose now that $\mathfrak{t}$ and $\mathfrak{q}$ are bounded. Then for each $X \in \mathscr{D}$ we have $X \in \mathfrak{t}[i]\cap\mathfrak{t}[j]^{\perp} \cap \mathfrak{q}[h] \cap \mathfrak{q}[k]^{\perp}$ for some $i,j,h,k \in \mathbb{Z}$ and thus $$X \in (\mathfrak{t} \cap \mathfrak{q})[\min \{i,j \}] \cap (\mathfrak{t} \cap \mathfrak{q})[\max \{h,k \}]^{\perp}$$
%as desired. 
%\end{proof}
%
%As an application, suppose that $(\mathfrak{t}[n],\mathfrak{q})$ is consistent for all $n \in \mathbb{Z}$ and denote $$\mathfrak{p}_n=\mathfrak{t}[n] \cap \mathfrak{q}$$
%By \hyperref[baa]{\textbf{Proposition \ref*{baa}}} $\mathfrak{p}_n$ is a t-structure and we have $$\mathfrak{p}_n[1] \subseteq \mathfrak{t}[n+1] \cap \mathfrak{q} = \mathfrak{p}_{n+1} \subseteq \mathfrak{p}_n$$ 
%We thus get a commutative (Hasse) diagram in $\mathfrak{ts}(\mathscr{D})$: 
%\begin{center}
%\begin{tikzcd}[ampersand replacement=\&]
% \vdots \arrow{d} \& \vdots \arrow{d} \& \vdots \arrow{d} \\
% \mathfrak{p}_{i-1}[-1] \arrow{d} \& \mathfrak{p}_{i-1} \arrow{d} \& \mathfrak{p}_{i-1}[1] \arrow{d} \\
%\mathfrak{p}_{i}[-1] \arrow{d} \arrow{ru} \& \mathfrak{p}_{i} \arrow{d} \arrow{ru}\& \mathfrak{p}_{i}[1] \arrow{d} \\
% \mathfrak{p}_{i+1}[-1] \arrow{d} \arrow{ru} \& \mathfrak{p}_{i+1} \arrow{d} \arrow{ru}\& \mathfrak{p}_{i+1}[1] \arrow{d} \\
% \vdots \& \vdots \& \vdots
%\end{tikzcd}
%\end{center}
%This will be useful later. \\

We conclude saying that that $\mathfrak{ts}(\mathscr{D})$ can be very big, especially  if $\mathscr{D}$ is not essentially small: for example, there is a proper class of t-structures on the derived categroy of abelian groups (see \cite{stan}).

\newpage

\subsection{The Grothendieck group} 
We will now introduce a first, and propbably most important, algebraic invariant for triangualted categories. The main reference is \cite{tom}, where the subject is developed in the direction of some 'Galois-style' results. Recall that the Grothendieck group of an abelian category $\mathscr{A}$ was introduced by Alexander Grothendieck in order to formulate his wonderful version of the Riemann-Roch theorem and is defined as the free abelian group on (isomorphism classes of) objects of $\mathscr{A}$ with a relation $B=A+C$ for each exact sequence $0 \longrightarrow A \longrightarrow B \longrightarrow C \longrightarrow 0$ in $\mathscr{A}$. We denote that group $K_0(\mathscr{A})$. Since the role of exact sequences in a triangulated category is played by distinguished triangles, we give the following definition. 

\begin{defn}
The \textbf{Grothendieck group} $K_0(\mathscr{D})$ of $\mathscr{D}$ is the free abelian group on (isomorphism classes of) objects of $\mathscr{D}$ with a relation $$Y=X+Z$$ for each distinguished triangle $X \longrightarrow Y \longrightarrow Z \longrightarrow$ in $\mathscr{D}$. 
\end{defn}

The Grothendieck group is functorial in the obvious way: a triangle functor $\mathscr{D} \longrightarrow \mathscr{D}'$ induces a homomorphism of groups between $K_0(\mathscr{D})$ and $K_0(\mathscr{D}')$, and similarly for cohomological functors (and abelian Grothendieck groups). \\
Now, by considering the very trivial triangle with zero vertices, the identity element $0 \in K_0(\mathscr{D})$ is the zero object $0 \in \mathscr{D}$, so there is no clash of notation. Since $X \longrightarrow X \oplus Y \longrightarrow Y \overset{0}{\longrightarrow} $ is distinguished, we have $X \oplus Y = X + Y$ for each $X,Y \in \mathscr{D}$. Moreover, by rotating multiple times the trivial triangle, we get $$X[n]=(-1)^nX$$ 
for each $X \in \mathscr{D}$, $n \in \mathbb{Z}$. We thus have that every element in $K_0(\mathscr{D})$ is represented by an object of $\mathscr{D}$, which doesn't happen in an abelian category. \\
%
%\begin{prop}
%Let $X,X' \in \mathscr{D}$. Then $X=X'$ in $K_0(\mathscr{D})$ if and only if there are two distinguished triangles in $\mathscr{D}$ of the form: $$E \longrightarrow X \oplus Y \longrightarrow Z \longrightarrow $$ $$E \longrightarrow X' \oplus Y \longrightarrow Z \longrightarrow $$
%\end{prop}
%
%\begin{proof}
%The 'if' part is obvious by the above computations. Now, denote $\sim$ the relation on objects of $\mathscr{D}$ involved in the statement of the proposition. If $X \longrightarrow Y \longrightarrow Z \longrightarrow $ is a distinguished triangle in $\mathscr{D}$, considering it with the distinguished triangle $X \longrightarrow X \oplus Z \longrightarrow Z \overset{0}{\longrightarrow}$ gives $X \oplus Z \sim Y$ . Now, if $X,X' \sim 0$, by definition we have four distinguished triangles $E \longrightarrow X \oplus Y \longrightarrow Z \longrightarrow$, $E \longrightarrow Y \longrightarrow Y \longrightarrow$, $E' \longrightarrow X' \oplus Y' \longrightarrow Z' \longrightarrow$, $E' \longrightarrow Y' \longrightarrow Y' \longrightarrow$. By summing the first triangle with the third and the second with the fourth, we obtain $X \oplus X' \sim 0$. Moreover, by translating the first and the second triangles we obtain $X[1] \sim 0$. This means that $\sim$ is a congruence relation on $K_0(\mathscr{D})$, which conicides with the relations of the Grothendicek group.
%\end{proof}
%
%Now, let $k$ be a field and suppose that $\mathscr{D}$ is $k$-linear.
%
%\begin{defn}
%$\mathscr{D}$ is \textbf{Hom-finite} if for each $X,Y \in \mathscr{D}$ the $k$-vector space $$\textnormal{Hom}_{\mathscr{D}}^{\bullet}(X,Y)=\bigoplus_{i \in \mathbb{Z}} \textnormal{Hom}_{\mathscr{D}}(X,Y[i])$$ is finite dimensional. If this case, we define the \textbf{Euler pairing} $$\chi(X,Y)= \sum_{i \in \mathbb{Z}} (-1)^i \textnormal{dim}_k(\textnormal{Hom}_{\mathscr{D}}(X,Y[i]))$$
%\end{defn}
%
%In other words, in the Hom-finite case, we have a bifuntor with values in the category of finite-dimensional $\mathbb{Z}$-graded vector spaces: $$\mathscr{D}^{\textnormal{op}} \times \mathscr{D} \xrightarrow{\textnormal{Hom}_{\mathscr{D}}^{\bullet}} \textnormal{grVect}_{k}^{\textnormal{fd}}$$
%Since the Hom-functors are cohomological, one obtains by applying functoriality of the Grothendieck group a bilinear map $$K_0(\mathscr{D}) \otimes_{\mathbb{Z}} K_0(\mathscr{D}) \longrightarrow K_0( \textnormal{grVect}_{k}^{\textnormal{fd}})=\mathbb{Z}^{\oplus \mathbb{Z}}$$
%and composing with the alternate sum morphism $\mathbb{Z}^{\oplus \mathbb{Z}} \longrightarrow \mathbb{Z}$ one obtains that $\chi$ is well-defined as a $\mathbb{Z}$-bilinear form on the Grothendieck group. \\
%
%\begin{defn}
%If $\mathscr{D}$ is Hom-finite, the \textbf{numerical Grothendieck group} of $\mathscr{D}$ is $$K_0^{\textnormal{nu}}(\mathscr{D})=K_0(\mathscr{D})/ \textnormal{ker}(\chi)$$
%where the $\textnormal{ker}(\chi)$ is the bikernel of $\chi$ (the intersection of its right kernel and the left one). If $K_0^{\textnormal{nu}}(\mathscr{D})$ has finite rank (as an abelian group), we say that $\mathscr{D}$ is \textbf{numerically finite}. 
%\end{defn}
%
%Observe that the definition immediately implies that the numerical Grothendieck group is torsion-free.  \\
%
%\Coffeecup \ We introduced the numerical Grothendieck group because the starting one was too big. For example, the Hirezbruch-Riemann-Roch theorem tells us that if $M$ is a compact complex manifold, then $\mathscr{D}^{\textnormal{b}}(\textnormal{Coh}(M))$ is numerically finite. 
%

\begin{exmp}
The Grothendieck group is not always interesting. As shown in \cite{miy}, if $R$ is an artinian (commutative and unitary) ring then $$K_0(\mathscr{D}(\textnormal{Mod}_R^{\textnormal{fin}}))=0$$
However, this doesn't happen for the bounded derived category (see \hyperref[fit]{\textbf{Example \ref*{fit}}})!
\end{exmp}

Let $\mathfrak{t}$ be a bounded t-structure on $\mathscr{D}$. The existence of Postnikov towers shows that for each $X \in \mathscr{D}$ $$X= \sum_{n \in \mathbb{Z}} H^n_{\mathfrak{t}}(X)$$ in $K_0(\mathscr{D})$ and, since $H^n_{\mathfrak{t}}(X) \in \heartsuit_{\mathfrak{t}}[n]$, we get that $X$ is an alternate sum of objects in $\heartsuit_{\mathfrak{t}}$. \\

\begin{prop} 
Let $\mathfrak{t}$ be a bounded t-structure on $\mathscr{D}$. Then the inclusion $\heartsuit_{\mathfrak{t}} \subseteq \mathscr{D}$ induces an isomorphism of groups $$K_0(\heartsuit_{\mathfrak{t}})=K_0(\mathscr{D})$$
\end{prop}

\begin{proof} 
Since exact sequences in $\heartsuit_{\mathfrak{t}}$ correspond to distinguished triangles in $\mathscr{D}$ with vertices in $\heartsuit_{\mathfrak{t}}$, the inclusion defines a morphism from $K_0(\heartsuit_{\mathfrak{t}})$ to $K_0(\mathscr{D})$. The inverse of this map is given by the alternate sum above. 
\end{proof}

\begin{exmp}\label{fit}
Let $k$ be a field. It is well-known that $K_0(\textnormal{Mod}_k^{\textnormal{fin}})=\mathbb{Z}$ and thus the same equality holds for the bounded derived category of finite dimensional $k$-vector spaces. The natural map $\mathscr{D}^b(\textnormal{Mod}_k^{\textnormal{fin}}) \longrightarrow K_0(\textnormal{Mod}_k^{\textnormal{fin}})=\mathbb{Z}$ is then the usual Euler characteristic, which is where it all began. 
\end{exmp}
