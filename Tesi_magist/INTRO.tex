\section{Ouverture}
\epigraph{\textit{Professor Soergel studies the fact that the sphere is round, and he likes it.}}{???}
\vspace{10 mm}
One manifestly challenging and highly creative task is to capture the geometry of some space (whatever the latter means) via clear and clean information. An historically notable breakthrough in that direction is probably the discovery by Euler of his famous formula for convex polyhedra, which we can regard as the date of birth of homological algebra. It was clear since then that a space should be regarded as a numerical (or, more generally, algebraic) object sliced into pieces, each describing the features appearing at a given dimensional level. These pieces interact by means of a 'boundary' map forming what we call today a 'chain complex' (i.e., an objects of a derived category), whose '(co)homology' measures how distant is the geometry from being quite trivial (having no holes, for example). However, there is also another interaction appearing when considering what's happening above a fixed dimension and not exactly in it; one obtains different maps arranged in a sequence called 'Postnikov tower', in honour of the well known parallel behavior arising in topology. The scope of this thesis is indeed to convince the reader that these towers synthesize most of the foundational aspects of homological algebra and allow us to understand fractal noninteger dimensions by slicing (algebrisations of) spaces into finer pieces. \\

We shall now get more in detail. As mentioned above, by (co)homology we mean the well known functor $$\mathscr{D}^b(\mathscr{A}) \overset{H^n}{\longrightarrow} \mathscr{A}$$ from the (bounded) derived category of some abelian category $\mathscr{A}$ to $\mathscr{A}$ itself. We shall expand our classical view of cohomology by first regarding these data from a different perspective. Truncating a complex in $\mathscr{D}^b(\mathscr{A})$ at the indices in which its cohomology doesn't vanish defines a factorization of the initial morphism, the abovementioned Postnikov tower of the complex, whose successive cones are, up to translation, the cohomology of the complex (with decreasing indices) and thus lie in $\mathscr{A}[n]$ for some $n \in \mathbb{Z}$ (where we think of $\mathscr{A}$ sitting in its derived category as the complexes concentrated in degree $0$). We then have a collection $\{ \mathscr{A}[n] \}_{n \in \mathbb{Z}}$ of subcategories indexed by integers which induces Postikov towers. This, with the obvious vanishing property of negative Ext groups, is the celebrated concept of 'bounded t-structure' ('t' stands for 'truncation') by Beilinson, Bernstein and Deligne. We can now think of a bounded t-structure (which makes sense in abstract triangulated categories) as a generalized cohomology theory: indeed, the abelianity of the subcategories we consider (which are the same up to translation) and the functoriality of the cones of the Postnikov towers (which we think of as the cohomology induced by the bounded t-structure) are automatic. A triangulated category tipically contains a lot of bounded t-structures different from the 'standard' one we described above, the most famous (and firstly discovered) being the one of 'perverse sheaves' on a stratified space. \\

By going deeper, a crucial motivation can be found in the ideas of David Mumford. We consider a compact Riemann surface $C$. For a coherent sheaf $F$ we can define its phase as $$- \frac{\textnormal{rk}(F)}{\textnormal{deg}(F)} \in \overline{\mathbb{Q}}=\mathbb{Q} \sqcup \{ \infty \}$$ 
We call a sheaf semistable if all its subsheaves have smaller phase and denote $\mathscr{P}_{\phi}$ the full subcategory of semistable sheaves of phase $\phi$. Now we again have a collection $\{ \mathscr{P}_{\phi} \}_{\phi \in \overline{\mathbb{Q}}}$ of subcategories of $\textnormal{Coh}(C)$ with a similar vanishing property (there is no nonzero morphism between two semistable sheaves if the first one has strictly greater phase). Moreover, each coherent sheaf has a filtration by subobjects, called 'Harder-Narasimhan filtration', whose successive quotients are semistable with decreasing phase. Combining these filtrations with the Postnikov towers of the standard t-structure on $\mathscr{D}^b(\textnormal{Coh}(C))$, one gets a collection $$\{ \mathscr{P}_{\phi}[n] \}_{(n,\phi) \in \mathbb{Z} \times \overline{\mathbb{Q}}}$$ of subcategories which has its own Postnikov towers with decreasing indices with respect to the lexicographical product on $\mathbb{Z} \times \overline{\mathbb{Q}}$. Now, if we choose a monotne embedding of $\overline{\mathbb{Q}}$ into the interval $]0,1] \subseteq \mathbb{R}$, then we get an embedding $$\mathbb{Z} \times \overline{\mathbb{Q}} \hookrightarrow \mathbb{Z} \times ]0,1] = \mathbb{R}$$ 
and we can then think of our subcategories as indexed by real numbers. These kind of phenomena also appear in string theory as '$\Pi$-stability': physicists think of $D$-branes as coherent sheaves on a Calabi-Yau manifold, and systems of (anti-)$D$-branes linked by tachyons as complexes of sheaves. Then a phase of such a system is a real number (the argument of its central charge precisely) that counts its symmetries and, by defining semistable branes just as in Mumford stability, one has an approximation (or 'decay', in a more concrete language) of each object in the derived category by semistables of decreasing phase. In other words, one gets Postnikov towers again.  \\

All this motivated Tom Bridgeland to define an $\mathbb{R}$-slicing in a triangulated category as a collection of subcategories indexed by real numbers with the above properties (vanishing and existence of Postnikov towers), generalizing both bounded t-structures and Mumford stability. Those $\mathbb{R}$-slicings can finally be regarded as cohomology theories where dimension is allowed to be real (intead of just integer). Thinking of the above example from string theory, this point of view reminds of Poincare's philosophy: dimension is just a shadow of simmetry, and thus it is natural to index both cohomology and simmetries by the same set. However, from a formal perspective, Bridgeland's approach is slightly redundant: both the main definition and its basic properties don't depend on any special feature of $\mathbb{R}$, but just on its ordering and the action it carries by $\mathbb{Z}$. Following Gorodentsev, Kuleshov and Rudakov, this leads us to introduce the notion of $\mathbb{Z}$-poset and consider $J$-slicings, where $J$ is a $\mathbb{Z}$-poset, as an immediate generalization of Bridgeland's ones. In this unified language, many things appear clearer: the set of all $J$-slicings on a fixed triangulated category is functorial in $J$ and this gives rise to a very natural approach to tilting theory by Happel, Reiten and Smal\o{} and can be exploited to recover classical constructions of nonstandard bounded t-structures like perverse sheaves and the diagonal t-structure from Koszul duality in a simple way.\\ 

Moreover, the formalism of $\mathbb{R}$-slicings doesn't explain the theory of semiorthogonal decompositions: these are important objects which appear in many areas of literature (and are intesively used even in the the context of Bridgeland stability) and arise in our language as $J$-slicings when $\mathbb{Z}$ acts trivially on $J$. Such $J$'s clearly don't map to $\mathbb{R}$ as $\mathbb{Z}$-posets, and thus semiorthogonal decompositions cannot be reconducted to Bridgeland's slicings. Another reason to pursue such a generality is the fact that Mumford stability extends to higher dimensional varieties as Gieseker stability, where the phase of a sheaf becomes its whole Hilbert function. This means that in this case $J$ should be taken as a set of polynomial germs with some natural ordering, and $\mathbb{R}$ is then again not sufficient. \\

Needless to say, the language of categories will serve us as a main tool. Besides the examples, which will be taken from all around, this thesis is purely categorical. The main backround required to the reader is some knowledge of algebraic categories (abelian and their derived ones) and basics from order theory. \\ \\

%\subsection{Ringraziamenti} Per difesa o per amore
